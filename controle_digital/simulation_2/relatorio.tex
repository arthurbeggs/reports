\documentclass{article}
\usepackage{indentfirst}
\usepackage[utf8]{inputenc}
\usepackage[T1]{fontenc}
\usepackage[brazilian]{babel}
\usepackage{lmodern}
\usepackage{graphicx}
\usepackage{float}
\usepackage[]{subfigure}
\usepackage{afterpage}
\usepackage{amsmath}
\usepackage{textcomp,gensymb}
\usepackage{nameref}
\usepackage{accents}
\usepackage{listings}
\usepackage{color,soul}
\usepackage[margin=1in]{geometry}
\usepackage{steinmetz}

\PassOptionsToPackage{hyphens}{url}\usepackage{hyperref}
\hypersetup{
    breaklinks = true,
}
\urlstyle{same}
\newcommand{\ubar}[1]{\underaccent{\bar}{#1}}
\renewcommand\thesection{\arabic{section}$^a$}
\renewcommand\thesubsection{(\alph{subsection})}
\definecolor{dkgreen}{rgb}{0,0.6,0}
\definecolor{gray}{rgb}{0.5,0.5,0.5}
\definecolor{mauve}{rgb}{0.58,0,0.82}
\lstset{
    frame=tb,
    language=Matlab,
    aboveskip=3mm,
    belowskip=3mm,
    showstringspaces=false,
    basicstyle={\small\ttfamily},
    numbers=none,
    numberstyle=\tiny\color{gray},
    keywordstyle=\color{blue},
    commentstyle=\color{dkgreen},
    stringstyle=\color{mauve},
    breaklines=true,
    breakatwhitespace=true,
    tabsize=4
}

\title{Trabalho}
\author{Arthur Matos}
\date{2019}

\begin{document}
% capa
\begin{titlepage}
    \begin{center}
        \centering
        \includegraphics[width=.7\linewidth]{images/logo_unb.png}\\[0.5cm]
        {\large \textbf{Universidade de Brasília}}\\[0.2cm]
        {\large \textbf{Departamento de Engenharia Elétrica}}\\[0.2cm]
        {\large \textbf{Controle Digital}}\\[4.8cm]
        {\bf \huge {Exercício de Simulação 2}}\\[0.2cm]
        {\bf \large {}}
    \end{center}

    \vspace{5cm}
    \hspace{2cm} {\noindent \bf \large {Aluno:}}\\
    \vspace{0.8cm}
    \hspace{2.35cm} {\large Arthur de Matos Beggs --------------------------------- 12/0111098}\\[1cm]

    \begin{center}
        {\large Brasília}\\
        {\large 2$^{\ubar{\circ}}$/2019}
    \end{center}

\end{titlepage}
\clearpage
\setcounter{page}{2}
% \tableofcontents
\clearpage

% % Template de figura
% \begin{figure}[H]
%     \centering
%         \includegraphics[width=1\linewidth]{images/}
%         \caption{}\label{fig:}
% \end{figure}

% % Corpo do Relatório

\section{Discretize a função de transferência $$ G(s) = \frac{Y(s)}{U(s)} = \frac{1}{(s+2)(s+3)} $$ usando os métodos a seguir.}
\subsection*{Para cada caso, calcule a solução para $u(t) = 0$ $\forall\, t \geq 0$, $y(0) = 100$, $\dot{y}(0) = 0$, $T = 0.1s$ e compare no Matlab a solução exata a tempo contínuo com os resultados obtidos.}

\subsection{Transformada de $ G(s) $ com segurador de ordem zero em série:}


\subsection{Regra retangular para frente:}


\subsection{Regra retangular para trás:}


\subsection{Regra trapeizodal:}


\subsection{Mapeamento exato de pólos e zeros:}


\vspace{5cm}
\section{Considere uma planta com entrada $u(t)$ e saída $y(t)$, cuja função de transferência é dada por $$ G(s) = \frac{Y(s)}{U(s)} = \frac{k_p}{Js^2} $$}
\subsection*{Deseja-se utilizar um controlador com a estrutura $$ U(s) = \frac{bk_C}{a}U_c(s) - k_C\frac{s+b}{s+a}Y(s), $$ onde $U_C(s)$ é o sinal de referência. O polinômio característico de malha fechada deve ser $$ P(s) = (s+\omega_0)(s^2 + \omega_0s + \omega_0^2) = s^3 + 2\omega_0s^2 + 2\omega_0^2s + \omega_0^3. $$}
\subsection*{Para isso, basta escolher os parâmetros do controlador como $a = 2\omega_0$, $b = \omega_0/2$ e $k_C = 2\frac{J\omega_0^2}{k_P}$. Nesse caso, é possível verificar que o tempo de acomodação de 5\% do sistema a malha fechada para a entrada degrau é $t_s(5\%) = 5.52/\omega_0$. Quando necessário, utilize $\omega_0 = 1$.}

\subsection{Mostre que a função de transferência de malha fechada é dada por $$ \frac{Y(s)}{U_C(s)} = \frac{ (\frac{\omega_0^2}{2})\left( s + 2\omega_0\right) }{ s^3 + 2\omega_0s^2 + 2\omega_0^2s + \omega_0^3 }: $$}


\subsection{Obtenha o diagrama de Bode do sistema a malha fechada e determine a frequência de corte $\omega_c$:}


\subsection{Verifique que a ação de controle pode ser escrita como $$ U(s) = k_C \left( \frac b a U_C(s) - Y(s) + X(s) \right); $$ $$ X(s) = \frac{a-b}{s+a} Y(s). $$ Passando essas equações para o domínio do tempo e fazendo $\frac{dx(t)}{dt} \approx \frac{x(t+T)-x(t)}{T}$, onde $T$ é o período de amostragem, mostre que o controlador pode ser discretizado da seguinte forma $$ u[k] = k_C \left( \frac b a u_c[k] - y[k] + x[k] \right); $$ $$ x[k+1] = x[k] + T\left[ (a-b)y[k] - ax[k]\right] : $$}


\subsection{Implemente no Simulink esse sistema com controlador a tempo contínuo e com controlador discretizado com períodos de amostragem $T = 0.2/\omega_0$, $T = 0.5/\omega_0$ e $T = 1.08/\omega_0$:}


\subsection{Para cada período de amostragem, calcule a relação entre a frequência de amostragem e a frequência de corte do sistema a malha fechada $\omega_s/\omega_c$:}


\subsection{Para cada período de amostragem, compare a saída do sistema a malha fechada com controlador contínuo com a saída do sistema a malha fechada com controlador discretizado. O que se pode afirmar sobre o tempo de assentamento para os sistemas com controlador discretizado?}


\subsection{Para cada período de amostragem, compare a ação de controle do sistema a malha fechada com controlador contínuo com a saída do sistema a malha fechada com controlador discretizado:}


\subsection{O que pode ser concluído acerca da seleção da frquência de amostragem para a discretização de controladores a tempo contínuo?}



\end{document}

